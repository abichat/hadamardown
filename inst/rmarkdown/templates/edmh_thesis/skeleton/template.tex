% This is the Reed College LaTeX thesis template. Most of the work
% for the document class was done by Sam Noble (SN), as well as this
% template. Later comments etc. by Ben Salzberg (BTS). Additional
% restructuring and APA support by Jess Youngberg (JY).
% Your comments and suggestions are more than welcome; please email
% them to cus@reed.edu
%
% See http://web.reed.edu/cis/help/latex.html for help. There are a
% great bunch of help pages there, with notes on
% getting started, bibtex, etc. Go there and read it if you're not
% already familiar with LaTeX.
%
% Any line that starts with a percent symbol is a comment.
% They won't show up in the document, and are useful for notes
% to yourself and explaining commands.
% Commenting also removes a line from the document;
% very handy for troubleshooting problems. -BTS

% As far as I know, this follows the requirements laid out in
% the 2002-2003 Senior Handbook. Ask a librarian to check the
% document before binding. -SN

%%
%% Preamble
%%
% \documentclass{<something>} must begin each LaTeX document
\documentclass[12pt,twoside]{reedthesis}
% Packages are extensions to the basic LaTeX functions. Whatever you
% want to typeset, there is probably a package out there for it.
% Chemistry (chemtex), screenplays, you name it.
% Check out CTAN to see: http://www.ctan.org/
%%
\usepackage{graphicx,latexsym}
\usepackage{amsmath}
\usepackage{amssymb,amsthm}
\usepackage{longtable,booktabs,setspace}
\usepackage{chemarr} %% Useful for one reaction arrow, useless if you're not a chem major
\usepackage[hyphens]{url}
% Added by CII
\usepackage{hyperref}
\usepackage{lmodern}
\usepackage{float}
\floatplacement{figure}{H}
% End of CII addition
\usepackage{rotating}

% Next line commented out by CII
%%% \usepackage{natbib}
% Comment out the natbib line above and uncomment the following two lines to use the new
% biblatex-chicago style, for Chicago A. Also make some changes at the end where the
% bibliography is included.
%\usepackage{biblatex-chicago}
%\bibliography{thesis}


% Added by CII (Thanks, Hadley!)
% Use ref for internal links
\renewcommand{\hyperref}[2][???]{\autoref{#1}}
\def\chapterautorefname{Chapter}
\def\sectionautorefname{Section}
\def\subsectionautorefname{Subsection}
% End of CII addition

% Added by CII
\usepackage{caption}
\captionsetup{width=5in}
% End of CII addition

% \usepackage{times} % other fonts are available like times, bookman, charter, palatino

% Syntax highlighting #22
$if(highlighting-macros)$
  $highlighting-macros$
$endif$


% Added by CII
%%% Copied from knitr
%% maxwidth is the original width if it's less than linewidth
%% otherwise use linewidth (to make sure the graphics do not exceed the margin)
\makeatletter
\def\maxwidth{ %
  \ifdim\Gin@nat@width>\linewidth
    \linewidth
  \else
    \Gin@nat@width
  \fi
}
\makeatother

\renewcommand{\contentsname}{Table of Contents}
% End of CII addition

\setlength{\parskip}{0pt}

% Added by CII
$if(space_between_paragraphs)$
  %\setlength{\parskip}{\baselineskip}
  \usepackage[parfill]{parskip}
$endif$

\providecommand{\tightlist}{%
  \setlength{\itemsep}{0pt}\setlength{\parskip}{0pt}}

$for(header-includes)$
	$header-includes$
$endfor$
% End of CII addition
%%
%% End Preamble
%%
%

\begin{document}

\thispagestyle{empty}
\vspace{-2cm}

\voffset-10pt


%%%
%% premiere ligne avec logo légaux et numéro national de thèse
%% (modifier les symboles % en début de lignes
%%  pour obtenir la configuration souhaitée)
%%%
\noindent
\hspace*{-1cm}\hbox{\includegraphics[width=8.6cm]{logos/ed_edmh-h.jpg}}
\hfill
\hspace*{-0.4cm}{\small {\bf NNT : $NNT$}}
%%%
%% Numéro National de Thèse, à préciser lors du second dépôt
%% de la thèse post-soutenance
%%%
\hfill
%%%
%% logo Etablissement inscripteur
%% (modifier les symboles % en début de lignes
%% pour obtenir la configuration souhaitée, et ajuster
%% éventuellement la taille) En cas de cotutelle internationale,
%% mettre à la place le logo de l'université étrangère et rajouter
%% en bas de page le logo de l'établissement inscripteur
%%%
%\hbox{\includegraphics[width=2.3cm]{logoCentraleSupelec.jpg}}
%\hbox{\includegraphics[width=2cm]{logoHEC.jpg}}
%\hbox{\includegraphics[width=2cm]{logoENSAE.jpg}}
%\hbox{\includegraphics[width=2cm]{logoENSCachan.jpg}}
%\hbox{\includegraphics[width=1.5cm]{logoENS.jpg}}
%\hbox{\includegraphics[width=1.9cm]{logoENSTA.jpg}}
%\hbox{\includegraphics[width=2.7cm]{logoUPSud_UPSaclay.jpg}}
%\hbox{\includegraphics[width=2cm]{logoTelecom.jpg}}
\hbox{\includegraphics[width=$taille_logo_inscripteur$]{$logo_inscripteur$}}
%\hbox{\includegraphics[width=2.2cm]{logoUVSQ.jpg}}
%\hbox{\includegraphics[width=2.5cm]{logoX.jpg}}
\vspace{7mm}

\begin{center}
{\Large\bf THÈSE DE DOCTORAT}
\end{center}
\begin{center}
{de }
\end{center}
\begin{center}
 {\Large\sc l'Université Paris-Saclay}\\
  \vspace*{0.4cm}
École doctorale de mathématiques Hadamard (EDMH, ED 574)\\
 \vspace*{0.4cm}
%%%
%% nom établissement inscripteur
%%%
{\small \it Établissement d'inscription : }
$inscripteur$

%    Centrale-Supélec\\
%    Ecole polytechnique\\
%    Ecole normale supérieure\\
%    Ecole normale supérieure de Cachan\\
%    \'Ecole nationale de la statistique et de l'administration économique\\
%    \'Ecole nationale supérieure de techniques avancées\\
%    Ecole des hautes études commerciales de Paris\\
%    Telecom SudParis\\
%    Telecom ParisTech\\
%    Université d'Evry-Val d'Essonne\\
%    Université Paris-Sud\\
%   Université de Versailles Saint-Quentin-en-Yvelines\\
 \vspace*{0.2cm}
%%%
%% nom établissement(s) d'accueil, si différent du précédent
%%%
$if(accueil)$
{\small \it \'Etablissement d'accueil : }
$accueil$

%    AgroParisTech\\
%    Commissariat \`a l'énergie atomique et aux énergies alternatives\\
    % Institut des hautes études scientifiques\\
%    Institut national de la recherche agronomique\\
\vspace*{0.2cm}
$endif$
%%%
%% nom laboratoire(s) d'accueil
%%%
{\small \it Laboratoire d'accueil : }
$labo$
%  Centre de mathématiques appliquées de Polytechnique, UMR 7641 CNRS\\
%  Centre de mathématiques et de leurs applications, UMR 8536 CNRS\\
%  Centre de mathématiques Laurent Schwartz, UMR 7640 CNRS\\
%  Département de mathématiques et applications, UMR 8553 CNRS\\
%  ENSAE-X Centre d'économie, statistique et sociologie, UMR 9194 CNRS\\
%  Fédération de Mathématiques, FR 3487 CNRS\\
%  Groupement de recherche et d'études en gestion (GREGHEC), UMR 8071 CNRS\\
%  Institut de physique théorique de Saclay, URA 2306 CNRS\\
%Laboratoire Alexander Grothendieck (ERL 9216 CNRS)\\
%  Laboratoire de mathématiques d'Orsay, UMR 8628 CNRS\\
%  Laboratoire de mathématiques de Versailles, UMR 8100 CNRS\\
%  Laboratoire de mathématiques et modélisation d'\'Evry, UMR 8071 CNRS-INRA\\
%  Laboratoire traitement et communication de l'information, UMR 5141 CNRS\\
%  Mathématiques et informatique appliquées, UMR 518 INRA\\
%  Mathématiques et informatique appliquées du génome \`a l'environnement, UR 1404
%  Services répartis, architectures, modélisation, validation, administration des réseaux, UMR 5157 CNRS\\
%  Unité de mathématiques appliquées, ENSTA-CNRS-INRIA\\
\vspace*{0.2cm}
\end{center}




\begin{center}
{\it Spécialité de doctorat : }
%%%
%%  Choisir l'une des trois spécialités suivantes (soumis à
%%  accord du comité de direction de l'EDMH)
%%%
{\large $specialite$}
\end{center}

\vspace{5mm}

\begin{center}
{\large\bf $auteur$}
\end{center}

\vspace{3mm}

\begin{center}
{\Large $titre$}
%%%
%% éventuellement sur deux lignes
%%%
\end{center}


\vspace{10mm}

\noindent{\small \it Date de soutenance~: } $date$

\vspace{5mm}

\noindent
{\small \it Après avis des rapporteurs~: }
\begin{tabular}{l}
{\sc $rapporteur1$} ($institution_rapporteur1$)\vspace{1mm}  \\
{\sc $rapporteur2$} ($institution_rapporteur2$)\\
\end{tabular}

\vspace{8mm}

\noindent
{\small \it Jury de soutenance~: }
%%%
%% par ordre alphabétique des membres
%%
%% version avant soutenance à adapter (un jury peut contenir
%% tout ou partie des rapporteurs, tout au partie des codirecteurs de thèses,
%% voire des invités). Le président du jury est choisi par le jury en
%% son sein le jour de la soutenance, et est indiqué sur les exemplaires
%% post-soutenance
%% En cas d'absense d'un membre de jury prévu, les exemplaires post-thèse
%% ne doivent faire figurer que les présents.
%%%%
\begin{tabular}{ll}
{\sc   $jure_1$}&($institution_jure_1$) {\small $role_jure_1$}\vspace{1mm}\\
{\sc   $jure_2$}&($institution_jure_2$) {\small $role_jure_2$}\vspace{1mm}\\
{\sc   $jure_3$}&($institution_jure_3$) {\small $role_jure_3$}\vspace{1mm}\\
{\sc   $jure_4$}&($institution_jure_4$) {\small $role_jure_4$}\vspace{1mm}\\
{\sc   $jure_5$}&($institution_jure_5$) {\small $role_jure_5$}\vspace{1mm}\\
{\sc   $jure_6$}&($institution_jure_6$) {\small $role_jure_6$}\vspace{1mm}\\
\end{tabular}
% \begin{tabular}{ll}
% {\sc   Prénom NOM}&(Institution) {\small Rapporteur}\vspace{1mm}\\
% {\sc   Prénom NOM}&(Institution) {\small Codirecteur de thèse}\vspace{1mm}\\
% {\sc   Prénom NOM}&(Institution) {\small Examinateur}\vspace{1mm}\\
% {\sc   Prénom NOM}&(Institution) {\small Examinateur}\vspace{1mm}\\
% {\sc   Prénom NOM}&(Institution) {\small Codirecteur de thèse}\vspace{1mm}\\
% {\sc   Prénom NOM}&(Institution) {\small Invité}\vspace{1mm}\\
% \end{tabular}


\vfill
\noindent
\hbox{\includegraphics[width=2.2cm]{logos/logo_fmjh.jpg}}
\hfill
%%%
%% logo Etablissement d'accueil, si différent du précédent
%% (modifier les symboles % en début de lignes
%% pour obtenir la configuration souhaitée, et ajuster
%% éventuellement la taille)
%%%
$if(logo_accueil)$
\hbox{\includegraphics[width=$taille_logo_accueil$]{$logo_accueil$}}
\hfill
$endif$
%\hbox{\includegraphics[width=2cm]{logoIHES.jpg}}
%\hbox{\includegraphics[width=2cm]{logoIHES.jpg}}
%\hbox{\includegraphics[width=2.1cm]{logoINRA.jpg}}
%\hbox{\includegraphics[width=2.5cm]{logoAgro.jpg}}
%\hbox{\includegraphics[width=1.5cm]{logoCEA.jpg}}
%%%
%% logo laboratoire(s) d'accueil, s'il existe
%%%
\hbox{\includegraphics[width=$taille_logo_labo$]{$logo_labo$}}
%\hbox{\includegraphics[width=2cm]{logoCMAP.jpg}}
%\hbox{\includegraphics[width=2.4cm]{logoCMLA.jpg}}
%\hbox{\includegraphics[width=2cm]{logoCMLS.jpg}}
%\hbox{\includegraphics[width=4cm]{logoDMA.jpeg}}
%\hbox{\includegraphics[width=2.2cm]{logoGREHEC.jpg}}
%\hbox{\includegraphics[width=2.1cm]{logoIPhT.jpeg}}
%\hbox{\includegraphics[width=2cm]{logoLAG.jpg}}
%\hbox{\includegraphics[width=1.6cm]{logoLAMME.jpg}}
%\hbox{\includegraphics[width=3cm]{logoLMO.jpg}}
%\hbox{\includegraphics[width=3.4cm]{logoLMV.jpeg}}
%\hbox{\includegraphics[width=2.7cm]{logoMaIAGE.png}}
%\hbox{\includegraphics[width=2.5cm]{logoSAMOVAR.jpg}}
%\hbox{\includegraphics[width=2cm]{logoUMA.jpg}}
\hfill \includegraphics[width=1cm]{logos/pictoParis-Saclay.jpg}




\frontmatter % this stuff will be roman-numbered
\pagestyle{empty} % this removes page numbers from the frontmatter

$if(acknowledgements)$
  \begin{acknowledgements}
    $acknowledgements$
  \end{acknowledgements}
$endif$

$if(preface)$
  \begin{preface}
    $preface$
  \end{preface}
$endif$

$if(toc)$
  \hypersetup{linkcolor=$if(toccolor)$$toccolor$$else$black$endif$}
  \setcounter{tocdepth}{$toc-depth$}
  \tableofcontents
$endif$

$if(lot)$
  \listoftables
$endif$

$if(lof)$
  \listoffigures
$endif$

$if(abstract)$
  \begin{abstract}
    $abstract$
  \end{abstract}
$endif$

$if(dedication)$
  \begin{dedication}
    $dedication$
  \end{dedication}
$endif$

\mainmatter % here the regular arabic numbering starts
\pagestyle{fancyplain} % turns page numbering back on

$body$

%%%
%%
%% Quatrième de couverture (last cover page of the memoirs)
%%
%%%

\newpage
\pagestyle{empty}
\hbox{\includegraphics[width=8.6cm]{logos/ed_edmh-h.jpg}}


\bigskip
\noindent\fbox{\parbox{\textwidth}{
{\bf Titre : }  Titre en français

\medskip
{\bf Mots Clefs : }  Mettre de 3 à 6 mots clefs

\medskip
{\bf Résumé : }


\vspace{7cm}

}}

\bigskip
\noindent\fbox{\parbox{\textwidth}{
{\bf Title : }  Title in english

\medskip
{\bf Keys words : }  3 to 6  key words (in english)

\medskip
{\bf Abstract : }


\vspace{7cm}

}}

\vfill
\hfill \includegraphics[width=1cm]{logos/pictoParis-Saclay.jpg}
\end{document}

